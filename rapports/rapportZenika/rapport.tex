\documentclass[a4paper,francais,titlepage]{report}
\usepackage[utf8x]{inputenc}  
%% utf8x support des espaces insécables ' ' au lieu da la macro ~)

\usepackage[T1]{fontenc}
\usepackage[francais]{babel}
\usepackage{latexsym}
\usepackage{url}
\usepackage{graphicx}
\usepackage{enumerate}
\usepackage{lscape}   %% pour le mode paysage \begin{landscape}
\usepackage{hyperref}
\hypersetup{
    colorlinks,
    citecolor=black,
    filecolor=black,
    linkcolor=black,
    urlcolor=black
}

% continuité numérotation figures
\usepackage{chngcntr} 
\counterwithout{figure}{chapter}

% style général
\usepackage{fancyhdr}
\usepackage{vmargin}

\setmarginsrb{2.5cm}{2cm}{2.5cm}{1cm}{0.5cm}{1.5cm}{2cm}{2cm}
%1 est la marge gauche
%2 est la marge en haut
%3 est la marge droite
%4 est la marge en bas
%5 fixe la hauteur de l'entête
%6 fixe la distance entre l'entête et le texte
%7 fixe la hauteur du pied de page
%8 fixe la distance entre le texte et le pied de page

%%ressources
\graphicspath{{../ressources/}}

\begin{document}
\pagestyle{fancy}

%%%%%%%%% Première page: titre
\begin{titlepage}
 \begin{center}
	\vspace*{3.5cm}
	\Large \textbf{Gifts4YourFriends} \\
	\vspace{0.2cm}
	\small Rapport
	\vspace{2cm}

	\huge \textbf{Nuit de l'informatique} \\
	\vspace{0.3cm}
	\large {Défi Zenika}
	\vspace{1.5cm}
 \end{center}
 
 \begin{flushleft}
	\normalsize {\hspace{6cm}Responsables de l'événement: \\ 
				 \hspace{7.0cm} Blanc Xavier \\
				 \hspace{6.7cm} Bromberg David \\
				 \hspace{6.6cm} Fleury Emmanuel}
	\vspace{2cm}
 \end{flushleft}
 
   \begin{center}
   \includegraphics[scale=0.3]{logo_bx1.jpg}
   	\vfill 
   	\vspace{2cm}
	\normalsize {Rage Against The Turing Machine\\}
	\vspace{1.5cm} 
	{\today}.
 \end{center}
\end{titlepage}
%%%%%%%% Fin première page

\section{Présentation de SCRUM}
SCRUM est une méthode agile de développement et permet au petites équipes de gérer un projet en permettant de rester souple et ouvert fasse au changement des besoins du clients. Cette méthode est composé d'une analyse complète (backlog) et ensuite est découpé en sprint qui dure généralement trois, quatre semaines cependant pour tenir les deadlines du concours, nous avons fais des sprint de 4 heures. Chaque sprint est composé de plusieurs taches a implémenter puis tester.
\section{Répartition des taches}
Compte tenu du nombre de défis que nous avons relevé, nous avons profité du grand nombre de personnes dans l'équipe et dispatcher l'effectif au différentes parties des défis. Six personnes s'occupaient de mettre en place le web service et la base de données correspondante. Trois personnes etaient responsablent de l'interface Android trois autres de l'interface windows phone 7 et les trois dernières réalisaient le site avec google web toolkit. Une personne par 'sous-equipe' etait désigné en tant que scrum-master supléant. A chaque fin de sprint, un point était fait pour valider l'avancé des différentes tâches a exécuter et affecter les nouvelles taches aux différentes personnes.
Pour nous aider dans notre gestion de projet nous avons utilisé un tableau blanc et énormément de post-it.

photos...

\section{Bilan}
Nous avons bien mis en pratique cette methode scrum 

\end{document}	